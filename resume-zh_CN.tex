% !TEX TS-program = xelatex
% !TEX encoding = UTF-8 Unicode
% !Mode:: "TeX:UTF-8"

\documentclass{resume}
\usepackage{zh_CN-Adobefonts_external} % Simplified Chinese Support using external fonts (./fonts/zh_CN-Adobe/)
% \usepackage{NotoSansSC_external}
% \usepackage{NotoSerifCJKsc_external}
% \usepackage{zh_CN-Adobefonts_internal} % Simplified Chinese Support using system fonts
\usepackage{linespacing_fix} % disable extra space before next section
\usepackage{cite}

\begin{document}
\pagenumbering{gobble} % suppress displaying page number

\name{王贇阳}

\basicInfo{
  \email{sunnycloudyang@outlook.com} \textperiodcentered\ 
  \phone{(+86) 18811527221} \textperiodcentered\ 
  \github[SunnyCloudYang]{https://github.com/SunnyCloudYang}}
 
\section{\faGraduationCap\  教育背景}
\datedsubsection{\textbf{清华大学}, 北京}{2021 -- 至今}
\textit{在读本科生}\ 自动化系,预计 2025 年 6 月毕业

\section{\faUsers\ 项目经历}

\datedsubsection{\textbf{基于STM32的智能宿舍系统}}{2024年1月 -- 2024年2月}
\role{C/C++, React Native}{课程项目}
\begin{onehalfspacing}
使用STM32开发板实现智能宿舍系统
\begin{itemize}
  \item 实现了温湿度传感器、人体红外传感器、OLED显示屏等外设的驱动
  \item 完成了基于React Native的手机APP初步开发
  \item 实现了基于蓝牙的手机APP控制
\end{itemize}
\end{onehalfspacing}

\datedsubsection{\textbf{基于React和Hugo的外贸网站}}{2023年10月 -- 2024年1月}
\role{Javascript, React, Hugo}{个人项目}
\begin{onehalfspacing}
使用React和Hugo实现外贸公司官网
\begin{itemize}
  \item 使用React实现了前端页面内容
  \item 使用Hugo实现了静态网站框架
  \item Repo: \LinkTargetOn{https://github.com/trademall/trademall.github.io}
\end{itemize}
\end{onehalfspacing}

\datedsubsection{\textbf{指纹匹配与特征提取}}{2023年12月}
\role{MATLAB}{课程项目}
\begin{onehalfspacing}
使用MATLAB对指纹进行细节点提取与匹配
\begin{itemize}
  \item 实现指纹图像输入到特征提取与匹配全流程
  \item 实现PalmCode的提取与匹配
  \item 准确率达到95\%,处理时间在900ms以内
\end{itemize}
\end{onehalfspacing}

\datedsubsection{\textbf{基于卷积神经网络的头部特征分类}}{2023年10月}
\role{Python, PyTorch}{课程项目}
\begin{onehalfspacing}
使用PyTorch实现卷积神经网络对CelebA数据集进行多特征分类
\begin{itemize}
  \item 实现了CNN、VGG16等网络结构
  \item 在CelebA数据集上准确率超过90\%
\end{itemize}
\end{onehalfspacing}

\datedsubsection{\textbf{基于深度强化学习的机器人导航}}{2022年5月 -- 2022年9月}
\role{Python, PyTorch, OpenCV, ROS2}{比赛项目}
\begin{onehalfspacing}
使用深度强化学习算法解决机器人导航问题
\begin{itemize}
  \item 实现了DQN等算法,使用gRPC协议通讯
  \item 在真实环境中进行训练与测试
  \item 实现了机器人在未知环境中的导航与目标跟随
  \item 获得小米CyberDog机器狗开发大赛特等奖
\end{itemize}
\end{onehalfspacing}

% Reference Test
%\datedsubsection{\textbf{Paper Title\cite{zaharia2012resilient}}}{May. 2015}
%An xxx optimized for xxx\cite{verma2015large}
%\begin{itemize}
%  \item main contribution
%\end{itemize}

\section{\faCogs\ 技能}
% increase linespacing [parsep=0.5ex]
\begin{itemize}[parsep=0.5ex]
  \item 编程语言: C/C++, Python, Javascript, MATLAB
  \item 平台: Windows, Linux, Android, STM32
  \item 框架:NodeJS, React, React Native, Bootstrap, PyTorch, Hugo
\end{itemize}

\section{\faHeartO\ 获奖情况}
\datedline{\textit{二等奖},清华大学第六届人工智能挑战赛}{2023年5月}
\datedline{\textit{特等奖}, 清华大学首届机器狗开发大赛}{2022年9月}
\datedline{\textit{一等奖},清华大学第七届新生C语言大赛}{2022年3月}
\datedline{\textit{新生奖、优胜奖}, 清华大学第二十三届电子设计大赛}{2021年12月}

\section{\faInfo\ 其他}
% increase linespacing [parsep=0.5ex]
\begin{itemize}[parsep=0.5ex]
  \item 个人博客: https://sunnycloudyang.github.io/
  \item GitHub: https://github.com/SunnyCloudYang/
  \item 语言: 英语 - 熟练(CET-6 615)
  \item 个人爱好:书法(九级),吉他
\end{itemize}

%% Reference
%\newpage
%\bibliographystyle{IEEETran}
%\bibliography{mycite}
\end{document}
