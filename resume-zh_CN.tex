% !TEX TS-program = xelatex
% !TEX encoding = UTF-8 Unicode
% !Mode:: "TeX:UTF-8"

\documentclass{resume}
\usepackage{zh_CN-Adobefonts_external} % Simplified Chinese Support using external fonts (./fonts/zh_CN-Adobe/)
% \usepackage{NotoSansSC_external}
% \usepackage{NotoSerifCJKsc_external}
% \usepackage{zh_CN-Adobefonts_internal} % Simplified Chinese Support using system fonts
\usepackage{linespacing_fix} % disable extra space before next section
\usepackage{cite}
% reduce the space between subsections
\usepackage{titlesec}
\titlespacing*{\subsection}{0pt}{0.36\baselineskip}{0.2\baselineskip}

\begin{document}
\pagenumbering{gobble} % suppress displaying page number

\name{王贇阳}

\basicInfo{
  \email{sunnycloudyang@outlook.com} \textperiodcentered\ 
  \phone{(+86) 18811527221} \textperiodcentered\ 
  \github[SunnyCloudYang]{https://github.com/SunnyCloudYang}}
 
\section{\faGraduationCap\  教育背景}
\datedsubsection{\textbf{清华大学}, 北京}{2021年9月 -- 至今}
\textit{在读本科生}\ 自动化系,预计 2025 年 6 月毕业\\
相关课程:数字图像处理、人工智能原理、模式识别与机器学习、数据结构与算法、计算机网络

\section{\faUsers\ 项目经历}
\datedsubsection{\textbf{基于深度学习的部分指纹匹配}}{2024年1月 -- 2024年3月}
\role{Python, PyTorch}{实验室项目}
复现PFVNet,使用 spatial transformer network 和自注意力机制实现部分指纹与大指纹匹配
\begin{itemize}
  \item 使用PyTorch实现了PFVNet网络结构
  \item 在FVC2006数据集上准确率超过95\%,配对时间小于60ms
\end{itemize}

\datedsubsection{\textbf{基于React和Hugo的外贸网站}}{2023年10月 -- 2024年1月}
\role{Javascript, React, Hugo}{个人项目}
使用React和Hugo实现外贸公司官网
\begin{itemize}
  \item 使用Hugo搭建静态网站框架后通过React实现了用户界面与后台管理界面。
  \item Repo: \LinkTargetOn{https://github.com/trademall/trademall.github.io}
\end{itemize}

\datedsubsection{\textbf{基于卷积神经网络的头部特征分类}}{2023年10月}
\role{Python, PyTorch}{课程项目}
使用PyTorch实现卷积神经网络对CelebA数据集进行多特征分类
\begin{itemize}
  \item 实现了CNN、VGG16等网络结构进行横向比较
  \item 单特征准确率在CelebA数据集上超过90\%
\end{itemize}

\datedsubsection{\textbf{基于深度强化学习的机器人导航}}{2022年5月 -- 2022年9月}
\role{Python, PyTorch, OpenCV, ROS2}{比赛项目}
使用深度强化学习算法解决机器人导航问题
\begin{itemize}
  \item 使用DQN算法训练,gRPC协议通讯
  \item 实现了机器人在未知环境中的导航与目标跟随
\end{itemize}

% Reference Test
%\datedsubsection{\textbf{Paper Title\cite{zaharia2012resilient}}}{May. 2015}
%An xxx optimized for xxx\cite{verma2015large}
%\begin{itemize}
%  \item main contribution
%\end{itemize}

\section{\faCogs\ 技能}
% increase linespacing [parsep=0.5ex]
\begin{itemize}
  \item 编程语言: C/C++, Python, Javascript, MATLAB
  \item 平台: Windows, Linux, Android, STM32
  \item 框架:NodeJS, React, React Native, Bootstrap, PyTorch, Hugo
\end{itemize}

\section{\faHeartO\ 获奖情况}
\datedline{\textit{二等奖},清华大学第六届人工智能挑战赛}{2023年5月}
\datedline{\textit{特等奖}, 清华大学首届机器狗开发大赛}{2022年9月}
\datedline{\textit{一等奖},清华大学第七届新生C语言大赛}{2022年3月}
\datedline{\textit{新生奖、优胜奖}, 清华大学第二十三届电子设计大赛}{2021年12月}

\section{\faInfo\ 其他}
% increase linespacing [parsep=0.5ex]
\begin{itemize}
  \item 个人博客: https://sunnycloudyang.github.io/
  \item GitHub: https://github.com/SunnyCloudYang/
  \item 语言: 英语 - 熟练(CET-6 615)
\end{itemize}

%% Reference
%\newpage
%\bibliographystyle{IEEETran}
%\bibliography{mycite}
\end{document}
